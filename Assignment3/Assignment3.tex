% jupyter nbconvert --to pdf HW0.ipynb --template clean_report.tplx
% Default to the notebook output style

    


% Inherit from the specified cell style.




    
\documentclass[11pt]{article}

    
    
    \usepackage[T1]{fontenc}
    % Nicer default font (+ math font) than Computer Modern for most use cases
    \usepackage{mathpazo}

    % Basic figure setup, for now with no caption control since it's done
    % automatically by Pandoc (which extracts ![](path) syntax from Markdown).
    \usepackage{graphicx}
    % We will generate all images so they have a width \maxwidth. This means
    % that they will get their normal width if they fit onto the page, but
    % are scaled down if they would overflow the margins.
    \makeatletter
    \def\maxwidth{\ifdim\Gin@nat@width>\linewidth\linewidth
    \else\Gin@nat@width\fi}
    \makeatother
    \let\Oldincludegraphics\includegraphics
    % Set max figure width to be 80% of text width, for now hardcoded.
    \renewcommand{\includegraphics}[1]{\Oldincludegraphics[width=.8\maxwidth]{#1}}
    % Ensure that by default, figures have no caption (until we provide a
    % proper Figure object with a Caption API and a way to capture that
    % in the conversion process - todo).
    \usepackage{caption}
    \DeclareCaptionLabelFormat{nolabel}{}
    \captionsetup{labelformat=nolabel}

    \usepackage{adjustbox} % Used to constrain images to a maximum size 
    \usepackage{xcolor} % Allow colors to be defined
    \usepackage{enumerate} % Needed for markdown enumerations to work
    \usepackage{geometry} % Used to adjust the document margins
    \usepackage{amsmath} % Equations
    \usepackage{amssymb} % Equations
    \usepackage{textcomp} % defines textquotesingle
    % Hack from http://tex.stackexchange.com/a/47451/13684:
    \AtBeginDocument{%
        \def\PYZsq{\textquotesingle}% Upright quotes in Pygmentized code
    }
    \usepackage{upquote} % Upright quotes for verbatim code
    \usepackage{eurosym} % defines \euro
    \usepackage[mathletters]{ucs} % Extended unicode (utf-8) support
    \usepackage[utf8x]{inputenc} % Allow utf-8 characters in the tex document
    \usepackage{fancyvrb} % verbatim replacement that allows latex
    \usepackage{grffile} % extends the file name processing of package graphics 
                         % to support a larger range 
    % The hyperref package gives us a pdf with properly built
    % internal navigation ('pdf bookmarks' for the table of contents,
    % internal cross-reference links, web links for URLs, etc.)
    \usepackage{hyperref}
    \usepackage{longtable} % longtable support required by pandoc >1.10
    \usepackage{booktabs}  % table support for pandoc > 1.12.2
    \usepackage[inline]{enumitem} % IRkernel/repr support (it uses the enumerate* environment)
    \usepackage[normalem]{ulem} % ulem is needed to support strikethroughs (\sout)
                                % normalem makes italics be italics, not underlines
    

    
    
    % Colors for the hyperref package
    \definecolor{urlcolor}{rgb}{0,.145,.698}
    \definecolor{linkcolor}{rgb}{.71,0.21,0.01}
    \definecolor{citecolor}{rgb}{.12,.54,.11}

    % ANSI colors
    \definecolor{ansi-black}{HTML}{3E424D}
    \definecolor{ansi-black-intense}{HTML}{282C36}
    \definecolor{ansi-red}{HTML}{E75C58}
    \definecolor{ansi-red-intense}{HTML}{B22B31}
    \definecolor{ansi-green}{HTML}{00A250}
    \definecolor{ansi-green-intense}{HTML}{007427}
    \definecolor{ansi-yellow}{HTML}{DDB62B}
    \definecolor{ansi-yellow-intense}{HTML}{B27D12}
    \definecolor{ansi-blue}{HTML}{208FFB}
    \definecolor{ansi-blue-intense}{HTML}{0065CA}
    \definecolor{ansi-magenta}{HTML}{D160C4}
    \definecolor{ansi-magenta-intense}{HTML}{A03196}
    \definecolor{ansi-cyan}{HTML}{60C6C8}
    \definecolor{ansi-cyan-intense}{HTML}{258F8F}
    \definecolor{ansi-white}{HTML}{C5C1B4}
    \definecolor{ansi-white-intense}{HTML}{A1A6B2}

    % commands and environments needed by pandoc snippets
    % extracted from the output of `pandoc -s`
    \providecommand{\tightlist}{%
      \setlength{\itemsep}{0pt}\setlength{\parskip}{0pt}}
    \DefineVerbatimEnvironment{Highlighting}{Verbatim}{commandchars=\\\{\}}
    % Add ',fontsize=\small' for more characters per line
    \newenvironment{Shaded}{}{}
    \newcommand{\KeywordTok}[1]{\textcolor[rgb]{0.00,0.44,0.13}{\textbf{{#1}}}}
    \newcommand{\DataTypeTok}[1]{\textcolor[rgb]{0.56,0.13,0.00}{{#1}}}
    \newcommand{\DecValTok}[1]{\textcolor[rgb]{0.25,0.63,0.44}{{#1}}}
    \newcommand{\BaseNTok}[1]{\textcolor[rgb]{0.25,0.63,0.44}{{#1}}}
    \newcommand{\FloatTok}[1]{\textcolor[rgb]{0.25,0.63,0.44}{{#1}}}
    \newcommand{\CharTok}[1]{\textcolor[rgb]{0.25,0.44,0.63}{{#1}}}
    \newcommand{\StringTok}[1]{\textcolor[rgb]{0.25,0.44,0.63}{{#1}}}
    \newcommand{\CommentTok}[1]{\textcolor[rgb]{0.38,0.63,0.69}{\textit{{#1}}}}
    \newcommand{\OtherTok}[1]{\textcolor[rgb]{0.00,0.44,0.13}{{#1}}}
    \newcommand{\AlertTok}[1]{\textcolor[rgb]{1.00,0.00,0.00}{\textbf{{#1}}}}
    \newcommand{\FunctionTok}[1]{\textcolor[rgb]{0.02,0.16,0.49}{{#1}}}
    \newcommand{\RegionMarkerTok}[1]{{#1}}
    \newcommand{\ErrorTok}[1]{\textcolor[rgb]{1.00,0.00,0.00}{\textbf{{#1}}}}
    \newcommand{\NormalTok}[1]{{#1}}
    
    % Additional commands for more recent versions of Pandoc
    \newcommand{\ConstantTok}[1]{\textcolor[rgb]{0.53,0.00,0.00}{{#1}}}
    \newcommand{\SpecialCharTok}[1]{\textcolor[rgb]{0.25,0.44,0.63}{{#1}}}
    \newcommand{\VerbatimStringTok}[1]{\textcolor[rgb]{0.25,0.44,0.63}{{#1}}}
    \newcommand{\SpecialStringTok}[1]{\textcolor[rgb]{0.73,0.40,0.53}{{#1}}}
    \newcommand{\ImportTok}[1]{{#1}}
    \newcommand{\DocumentationTok}[1]{\textcolor[rgb]{0.73,0.13,0.13}{\textit{{#1}}}}
    \newcommand{\AnnotationTok}[1]{\textcolor[rgb]{0.38,0.63,0.69}{\textbf{\textit{{#1}}}}}
    \newcommand{\CommentVarTok}[1]{\textcolor[rgb]{0.38,0.63,0.69}{\textbf{\textit{{#1}}}}}
    \newcommand{\VariableTok}[1]{\textcolor[rgb]{0.10,0.09,0.49}{{#1}}}
    \newcommand{\ControlFlowTok}[1]{\textcolor[rgb]{0.00,0.44,0.13}{\textbf{{#1}}}}
    \newcommand{\OperatorTok}[1]{\textcolor[rgb]{0.40,0.40,0.40}{{#1}}}
    \newcommand{\BuiltInTok}[1]{{#1}}
    \newcommand{\ExtensionTok}[1]{{#1}}
    \newcommand{\PreprocessorTok}[1]{\textcolor[rgb]{0.74,0.48,0.00}{{#1}}}
    \newcommand{\AttributeTok}[1]{\textcolor[rgb]{0.49,0.56,0.16}{{#1}}}
    \newcommand{\InformationTok}[1]{\textcolor[rgb]{0.38,0.63,0.69}{\textbf{\textit{{#1}}}}}
    \newcommand{\WarningTok}[1]{\textcolor[rgb]{0.38,0.63,0.69}{\textbf{\textit{{#1}}}}}
    
    
    % Define a nice break command that doesn't care if a line doesn't already
    % exist.
    \def\br{\hspace*{\fill} \\* }
    % Math Jax compatability definitions
    \def\gt{>}
    \def\lt{<}
    % Document parameters
    
    \title{EE2703 Applied Programming Lab - Assignment 3}            

    
    
\author{
  \textbf{Name}: Rajat Vadiraj Dwaraknath\\
  \textbf{Roll Number}: EE16B033
}

    

    % Pygments definitions
    
\makeatletter
\def\PY@reset{\let\PY@it=\relax \let\PY@bf=\relax%
    \let\PY@ul=\relax \let\PY@tc=\relax%
    \let\PY@bc=\relax \let\PY@ff=\relax}
\def\PY@tok#1{\csname PY@tok@#1\endcsname}
\def\PY@toks#1+{\ifx\relax#1\empty\else%
    \PY@tok{#1}\expandafter\PY@toks\fi}
\def\PY@do#1{\PY@bc{\PY@tc{\PY@ul{%
    \PY@it{\PY@bf{\PY@ff{#1}}}}}}}
\def\PY#1#2{\PY@reset\PY@toks#1+\relax+\PY@do{#2}}

\expandafter\def\csname PY@tok@cm\endcsname{\let\PY@it=\textit\def\PY@tc##1{\textcolor[rgb]{0.25,0.50,0.50}{##1}}}
\expandafter\def\csname PY@tok@fm\endcsname{\def\PY@tc##1{\textcolor[rgb]{0.00,0.00,1.00}{##1}}}
\expandafter\def\csname PY@tok@dl\endcsname{\def\PY@tc##1{\textcolor[rgb]{0.73,0.13,0.13}{##1}}}
\expandafter\def\csname PY@tok@sd\endcsname{\let\PY@it=\textit\def\PY@tc##1{\textcolor[rgb]{0.73,0.13,0.13}{##1}}}
\expandafter\def\csname PY@tok@s2\endcsname{\def\PY@tc##1{\textcolor[rgb]{0.73,0.13,0.13}{##1}}}
\expandafter\def\csname PY@tok@c\endcsname{\let\PY@it=\textit\def\PY@tc##1{\textcolor[rgb]{0.25,0.50,0.50}{##1}}}
\expandafter\def\csname PY@tok@ch\endcsname{\let\PY@it=\textit\def\PY@tc##1{\textcolor[rgb]{0.25,0.50,0.50}{##1}}}
\expandafter\def\csname PY@tok@kt\endcsname{\def\PY@tc##1{\textcolor[rgb]{0.69,0.00,0.25}{##1}}}
\expandafter\def\csname PY@tok@nf\endcsname{\def\PY@tc##1{\textcolor[rgb]{0.00,0.00,1.00}{##1}}}
\expandafter\def\csname PY@tok@nv\endcsname{\def\PY@tc##1{\textcolor[rgb]{0.10,0.09,0.49}{##1}}}
\expandafter\def\csname PY@tok@gd\endcsname{\def\PY@tc##1{\textcolor[rgb]{0.63,0.00,0.00}{##1}}}
\expandafter\def\csname PY@tok@nt\endcsname{\let\PY@bf=\textbf\def\PY@tc##1{\textcolor[rgb]{0.00,0.50,0.00}{##1}}}
\expandafter\def\csname PY@tok@sb\endcsname{\def\PY@tc##1{\textcolor[rgb]{0.73,0.13,0.13}{##1}}}
\expandafter\def\csname PY@tok@si\endcsname{\let\PY@bf=\textbf\def\PY@tc##1{\textcolor[rgb]{0.73,0.40,0.53}{##1}}}
\expandafter\def\csname PY@tok@cs\endcsname{\let\PY@it=\textit\def\PY@tc##1{\textcolor[rgb]{0.25,0.50,0.50}{##1}}}
\expandafter\def\csname PY@tok@no\endcsname{\def\PY@tc##1{\textcolor[rgb]{0.53,0.00,0.00}{##1}}}
\expandafter\def\csname PY@tok@cp\endcsname{\def\PY@tc##1{\textcolor[rgb]{0.74,0.48,0.00}{##1}}}
\expandafter\def\csname PY@tok@gr\endcsname{\def\PY@tc##1{\textcolor[rgb]{1.00,0.00,0.00}{##1}}}
\expandafter\def\csname PY@tok@nc\endcsname{\let\PY@bf=\textbf\def\PY@tc##1{\textcolor[rgb]{0.00,0.00,1.00}{##1}}}
\expandafter\def\csname PY@tok@w\endcsname{\def\PY@tc##1{\textcolor[rgb]{0.73,0.73,0.73}{##1}}}
\expandafter\def\csname PY@tok@ss\endcsname{\def\PY@tc##1{\textcolor[rgb]{0.10,0.09,0.49}{##1}}}
\expandafter\def\csname PY@tok@mh\endcsname{\def\PY@tc##1{\textcolor[rgb]{0.40,0.40,0.40}{##1}}}
\expandafter\def\csname PY@tok@s\endcsname{\def\PY@tc##1{\textcolor[rgb]{0.73,0.13,0.13}{##1}}}
\expandafter\def\csname PY@tok@sr\endcsname{\def\PY@tc##1{\textcolor[rgb]{0.73,0.40,0.53}{##1}}}
\expandafter\def\csname PY@tok@s1\endcsname{\def\PY@tc##1{\textcolor[rgb]{0.73,0.13,0.13}{##1}}}
\expandafter\def\csname PY@tok@se\endcsname{\let\PY@bf=\textbf\def\PY@tc##1{\textcolor[rgb]{0.73,0.40,0.13}{##1}}}
\expandafter\def\csname PY@tok@kp\endcsname{\def\PY@tc##1{\textcolor[rgb]{0.00,0.50,0.00}{##1}}}
\expandafter\def\csname PY@tok@mi\endcsname{\def\PY@tc##1{\textcolor[rgb]{0.40,0.40,0.40}{##1}}}
\expandafter\def\csname PY@tok@na\endcsname{\def\PY@tc##1{\textcolor[rgb]{0.49,0.56,0.16}{##1}}}
\expandafter\def\csname PY@tok@gu\endcsname{\let\PY@bf=\textbf\def\PY@tc##1{\textcolor[rgb]{0.50,0.00,0.50}{##1}}}
\expandafter\def\csname PY@tok@cpf\endcsname{\let\PY@it=\textit\def\PY@tc##1{\textcolor[rgb]{0.25,0.50,0.50}{##1}}}
\expandafter\def\csname PY@tok@ge\endcsname{\let\PY@it=\textit}
\expandafter\def\csname PY@tok@kr\endcsname{\let\PY@bf=\textbf\def\PY@tc##1{\textcolor[rgb]{0.00,0.50,0.00}{##1}}}
\expandafter\def\csname PY@tok@sh\endcsname{\def\PY@tc##1{\textcolor[rgb]{0.73,0.13,0.13}{##1}}}
\expandafter\def\csname PY@tok@mf\endcsname{\def\PY@tc##1{\textcolor[rgb]{0.40,0.40,0.40}{##1}}}
\expandafter\def\csname PY@tok@kd\endcsname{\let\PY@bf=\textbf\def\PY@tc##1{\textcolor[rgb]{0.00,0.50,0.00}{##1}}}
\expandafter\def\csname PY@tok@ne\endcsname{\let\PY@bf=\textbf\def\PY@tc##1{\textcolor[rgb]{0.82,0.25,0.23}{##1}}}
\expandafter\def\csname PY@tok@mb\endcsname{\def\PY@tc##1{\textcolor[rgb]{0.40,0.40,0.40}{##1}}}
\expandafter\def\csname PY@tok@nn\endcsname{\let\PY@bf=\textbf\def\PY@tc##1{\textcolor[rgb]{0.00,0.00,1.00}{##1}}}
\expandafter\def\csname PY@tok@sa\endcsname{\def\PY@tc##1{\textcolor[rgb]{0.73,0.13,0.13}{##1}}}
\expandafter\def\csname PY@tok@il\endcsname{\def\PY@tc##1{\textcolor[rgb]{0.40,0.40,0.40}{##1}}}
\expandafter\def\csname PY@tok@vg\endcsname{\def\PY@tc##1{\textcolor[rgb]{0.10,0.09,0.49}{##1}}}
\expandafter\def\csname PY@tok@m\endcsname{\def\PY@tc##1{\textcolor[rgb]{0.40,0.40,0.40}{##1}}}
\expandafter\def\csname PY@tok@mo\endcsname{\def\PY@tc##1{\textcolor[rgb]{0.40,0.40,0.40}{##1}}}
\expandafter\def\csname PY@tok@k\endcsname{\let\PY@bf=\textbf\def\PY@tc##1{\textcolor[rgb]{0.00,0.50,0.00}{##1}}}
\expandafter\def\csname PY@tok@nl\endcsname{\def\PY@tc##1{\textcolor[rgb]{0.63,0.63,0.00}{##1}}}
\expandafter\def\csname PY@tok@gs\endcsname{\let\PY@bf=\textbf}
\expandafter\def\csname PY@tok@sc\endcsname{\def\PY@tc##1{\textcolor[rgb]{0.73,0.13,0.13}{##1}}}
\expandafter\def\csname PY@tok@sx\endcsname{\def\PY@tc##1{\textcolor[rgb]{0.00,0.50,0.00}{##1}}}
\expandafter\def\csname PY@tok@vc\endcsname{\def\PY@tc##1{\textcolor[rgb]{0.10,0.09,0.49}{##1}}}
\expandafter\def\csname PY@tok@nd\endcsname{\def\PY@tc##1{\textcolor[rgb]{0.67,0.13,1.00}{##1}}}
\expandafter\def\csname PY@tok@gi\endcsname{\def\PY@tc##1{\textcolor[rgb]{0.00,0.63,0.00}{##1}}}
\expandafter\def\csname PY@tok@c1\endcsname{\let\PY@it=\textit\def\PY@tc##1{\textcolor[rgb]{0.25,0.50,0.50}{##1}}}
\expandafter\def\csname PY@tok@ni\endcsname{\let\PY@bf=\textbf\def\PY@tc##1{\textcolor[rgb]{0.60,0.60,0.60}{##1}}}
\expandafter\def\csname PY@tok@nb\endcsname{\def\PY@tc##1{\textcolor[rgb]{0.00,0.50,0.00}{##1}}}
\expandafter\def\csname PY@tok@err\endcsname{\def\PY@bc##1{\setlength{\fboxsep}{0pt}\fcolorbox[rgb]{1.00,0.00,0.00}{1,1,1}{\strut ##1}}}
\expandafter\def\csname PY@tok@o\endcsname{\def\PY@tc##1{\textcolor[rgb]{0.40,0.40,0.40}{##1}}}
\expandafter\def\csname PY@tok@gt\endcsname{\def\PY@tc##1{\textcolor[rgb]{0.00,0.27,0.87}{##1}}}
\expandafter\def\csname PY@tok@ow\endcsname{\let\PY@bf=\textbf\def\PY@tc##1{\textcolor[rgb]{0.67,0.13,1.00}{##1}}}
\expandafter\def\csname PY@tok@kn\endcsname{\let\PY@bf=\textbf\def\PY@tc##1{\textcolor[rgb]{0.00,0.50,0.00}{##1}}}
\expandafter\def\csname PY@tok@kc\endcsname{\let\PY@bf=\textbf\def\PY@tc##1{\textcolor[rgb]{0.00,0.50,0.00}{##1}}}
\expandafter\def\csname PY@tok@vm\endcsname{\def\PY@tc##1{\textcolor[rgb]{0.10,0.09,0.49}{##1}}}
\expandafter\def\csname PY@tok@go\endcsname{\def\PY@tc##1{\textcolor[rgb]{0.53,0.53,0.53}{##1}}}
\expandafter\def\csname PY@tok@gp\endcsname{\let\PY@bf=\textbf\def\PY@tc##1{\textcolor[rgb]{0.00,0.00,0.50}{##1}}}
\expandafter\def\csname PY@tok@bp\endcsname{\def\PY@tc##1{\textcolor[rgb]{0.00,0.50,0.00}{##1}}}
\expandafter\def\csname PY@tok@gh\endcsname{\let\PY@bf=\textbf\def\PY@tc##1{\textcolor[rgb]{0.00,0.00,0.50}{##1}}}
\expandafter\def\csname PY@tok@vi\endcsname{\def\PY@tc##1{\textcolor[rgb]{0.10,0.09,0.49}{##1}}}

\def\PYZbs{\char`\\}
\def\PYZus{\char`\_}
\def\PYZob{\char`\{}
\def\PYZcb{\char`\}}
\def\PYZca{\char`\^}
\def\PYZam{\char`\&}
\def\PYZlt{\char`\<}
\def\PYZgt{\char`\>}
\def\PYZsh{\char`\#}
\def\PYZpc{\char`\%}
\def\PYZdl{\char`\$}
\def\PYZhy{\char`\-}
\def\PYZsq{\char`\'}
\def\PYZdq{\char`\"}
\def\PYZti{\char`\~}
% for compatibility with earlier versions
\def\PYZat{@}
\def\PYZlb{[}
\def\PYZrb{]}
\makeatother


    % Exact colors from NB
    \definecolor{incolor}{rgb}{0.0, 0.0, 0.5}
    \definecolor{outcolor}{rgb}{0.545, 0.0, 0.0}



    
    % Prevent overflowing lines due to hard-to-break entities
    \sloppy 
    % Setup hyperref package
    \hypersetup{
      breaklinks=true,  % so long urls are correctly broken across lines
      colorlinks=true,
      urlcolor=urlcolor,
      linkcolor=linkcolor,
      citecolor=citecolor,
      }
    % Slightly bigger margins than the latex defaults
    
    \geometry{verbose,tmargin=1in,bmargin=1in,lmargin=1in,rmargin=1in}
    
    

    \begin{document}
    
    
    \maketitle
    
    

    
	

	
		
    \section{Introduction}\label{introduction}

In this assignment, an analysis of fourier series approximations on two
functions is done, namely \(e^x\) and \(\cos(\cos(x))\). The basic idea
of fourier series is that certain well behaved functions can be
represented as an infinte series of scaled harmonics. The coefficients
which are used to scale each harmonic are found using the integration
formulas given in the assignment text. The fourier coefficients for the
above two functions are estimated by numerically evaluating these
integrals. Another approach for estimating the fourier coefficients is
also done, namely least squares estimation. The results of the two
approaches are compared. The differences in convergence between the two
test functions are also noted.

\section{Part 1}\label{part-1}

Numpy and matplotlib are imported inline using pylab. The plot size and
font size are increased.

	

	
		
    \begin{Verbatim}[commandchars=\\\{\}]
{\color{incolor}In [{\color{incolor}1}]:} \PY{c+c1}{\PYZsh{} Importing numpy and matplotlib}
        \PY{o}{\PYZpc{}}\PY{k}{pylab} inline
        \PY{k+kn}{from} \PY{n+nn}{scipy}\PY{n+nn}{.}\PY{n+nn}{integrate} \PY{k}{import} \PY{n}{quad}
\end{Verbatim}


    \begin{Verbatim}[commandchars=\\\{\}]
Populating the interactive namespace from numpy and matplotlib

    \end{Verbatim}

	

	
		
    \begin{Verbatim}[commandchars=\\\{\}]
{\color{incolor}In [{\color{incolor}2}]:} \PY{c+c1}{\PYZsh{} Increase figure and font size}
        \PY{n}{rcParams}\PY{p}{[}\PY{l+s+s1}{\PYZsq{}}\PY{l+s+s1}{figure.figsize}\PY{l+s+s1}{\PYZsq{}}\PY{p}{]} \PY{o}{=} \PY{l+m+mi}{12}\PY{p}{,}\PY{l+m+mi}{9}
        \PY{n}{rcParams}\PY{p}{[}\PY{l+s+s1}{\PYZsq{}}\PY{l+s+s1}{font.size}\PY{l+s+s1}{\PYZsq{}}\PY{p}{]} \PY{o}{=} \PY{l+m+mi}{18}
        \PY{n}{rcParams}\PY{p}{[}\PY{l+s+s1}{\PYZsq{}}\PY{l+s+s1}{text.usetex}\PY{l+s+s1}{\PYZsq{}}\PY{p}{]} \PY{o}{=} \PY{k+kc}{True}
\end{Verbatim}


	

	
		
    \begin{Verbatim}[commandchars=\\\{\}]
{\color{incolor}In [{\color{incolor}3}]:} \PY{k}{def} \PY{n+nf}{periodic}\PY{p}{(}\PY{n}{a}\PY{p}{,} \PY{n}{b}\PY{p}{)}\PY{p}{:}
            \PY{l+s+sd}{\PYZdq{}\PYZdq{}\PYZdq{}}
        \PY{l+s+sd}{    Decorator to periodically extend a function }
        \PY{l+s+sd}{    using the values in the interval (a,b).}
        \PY{l+s+sd}{    \PYZdq{}\PYZdq{}\PYZdq{}}
            \PY{n}{interval} \PY{o}{=} \PY{n}{b} \PY{o}{\PYZhy{}} \PY{n}{a}
            \PY{k}{return} \PY{k}{lambda} \PY{n}{f}\PY{p}{:} \PY{k}{lambda} \PY{n}{x}\PY{p}{:} \PY{n}{f}\PY{p}{(}\PY{p}{(}\PY{n}{x} \PY{o}{\PYZhy{}} \PY{n}{a}\PY{p}{)} \PY{o}{\PYZpc{}} \PY{n}{interval} \PY{o}{+} \PY{n}{a}\PY{p}{)}
\end{Verbatim}


	

	
		
    The periodic extensions of the two functions over an interval
\([0,2\pi]\) are created below:

	

	
		
    \begin{Verbatim}[commandchars=\\\{\}]
{\color{incolor}In [{\color{incolor}4}]:} \PY{n+nd}{@periodic}\PY{p}{(}\PY{l+m+mi}{0}\PY{p}{,}\PY{l+m+mi}{2}\PY{o}{*}\PY{n}{pi}\PY{p}{)}
        \PY{k}{def} \PY{n+nf}{coscos}\PY{p}{(}\PY{n}{x}\PY{p}{)}\PY{p}{:}
            \PY{k}{return} \PY{n}{cos}\PY{p}{(}\PY{n}{cos}\PY{p}{(}\PY{n}{x}\PY{p}{)}\PY{p}{)}
        
        \PY{n+nd}{@periodic}\PY{p}{(}\PY{l+m+mi}{0}\PY{p}{,}\PY{l+m+mi}{2}\PY{o}{*}\PY{n}{pi}\PY{p}{)}
        \PY{k}{def} \PY{n+nf}{per\PYZus{}e}\PY{p}{(}\PY{n}{x}\PY{p}{)}\PY{p}{:}
            \PY{k}{return} \PY{n}{exp}\PY{p}{(}\PY{n}{x}\PY{p}{)}
\end{Verbatim}


	

	
		
    \begin{Verbatim}[commandchars=\\\{\}]
{\color{incolor}In [{\color{incolor}5}]:} \PY{n}{x} \PY{o}{=} \PY{n}{linspace}\PY{p}{(}\PY{o}{\PYZhy{}}\PY{l+m+mi}{2}\PY{o}{*}\PY{n}{pi}\PY{p}{,}\PY{l+m+mi}{4}\PY{o}{*}\PY{n}{pi}\PY{p}{,}\PY{l+m+mi}{400}\PY{p}{)}
        \PY{n}{coscos}\PY{p}{(}\PY{n}{arange}\PY{p}{(}\PY{l+m+mi}{10}\PY{p}{)}\PY{p}{)} \PY{c+c1}{\PYZsh{} Works with numpy arrays}
\end{Verbatim}


\begin{Verbatim}[commandchars=\\\{\}]
{\color{outcolor}Out[{\color{outcolor}5}]:} array([ 0.54030231,  0.85755322,  0.91465333,  0.54869613,  0.79387345,
                0.96003693,  0.57338048,  0.72902338,  0.98943353,  0.61285301])
\end{Verbatim}
            
	

	
		
    The functions and their periodic versions are plotted below. The fourier
series will yield the periodic version of the function as its result.

	

	
		
    \begin{Verbatim}[commandchars=\\\{\}]
{\color{incolor}In [{\color{incolor}6}]:} \PY{n}{fig1} \PY{o}{=} \PY{n}{figure}\PY{p}{(}\PY{p}{)}
        \PY{n}{ax} \PY{o}{=} \PY{n}{fig1}\PY{o}{.}\PY{n}{add\PYZus{}subplot}\PY{p}{(}\PY{l+m+mi}{1}\PY{p}{,}\PY{l+m+mi}{1}\PY{p}{,}\PY{l+m+mi}{1}\PY{p}{)}
        \PY{n}{ax}\PY{o}{.}\PY{n}{set\PYZus{}yscale}\PY{p}{(}\PY{l+s+s1}{\PYZsq{}}\PY{l+s+s1}{log}\PY{l+s+s1}{\PYZsq{}}\PY{p}{)}
        \PY{n}{grid}\PY{p}{(}\PY{k+kc}{True}\PY{p}{)}
        \PY{n}{plot}\PY{p}{(}\PY{n}{x}\PY{p}{,}\PY{n}{exp}\PY{p}{(}\PY{n}{x}\PY{p}{)}\PY{p}{)}
        \PY{n}{plot}\PY{p}{(}\PY{n}{x}\PY{p}{,}\PY{n}{per\PYZus{}e}\PY{p}{(}\PY{n}{x}\PY{p}{)}\PY{p}{)}
        \PY{n}{title}\PY{p}{(}\PY{l+s+s2}{\PYZdq{}}\PY{l+s+s2}{\PYZdl{}e\PYZca{}x\PYZdl{} and its periodically extended version}\PY{l+s+s2}{\PYZdq{}}\PY{p}{)}
        \PY{n}{ylabel}\PY{p}{(}\PY{l+s+sa}{r}\PY{l+s+s2}{\PYZdq{}}\PY{l+s+s2}{\PYZdl{}y\PYZdl{} (log)}\PY{l+s+s2}{\PYZdq{}}\PY{p}{,} \PY{n}{fontsize} \PY{o}{=} \PY{l+m+mi}{26}\PY{p}{)}
        \PY{n}{xlabel}\PY{p}{(}\PY{l+s+sa}{r}\PY{l+s+s2}{\PYZdq{}}\PY{l+s+s2}{\PYZdl{}x\PYZdl{} (linear)}\PY{l+s+s2}{\PYZdq{}}\PY{p}{)}
        \PY{n}{legend}\PY{p}{(}\PY{p}{[}\PY{l+s+sa}{r}\PY{l+s+s2}{\PYZdq{}}\PY{l+s+s2}{\PYZdl{}e\PYZca{}x\PYZdl{}}\PY{l+s+s2}{\PYZdq{}}\PY{p}{,}\PY{l+s+sa}{r}\PY{l+s+s2}{\PYZdq{}}\PY{l+s+s2}{periodic version of \PYZdl{}e\PYZca{}x\PYZdl{}}\PY{l+s+s2}{\PYZdq{}}\PY{p}{]}\PY{p}{,} \PY{n}{loc}\PY{o}{=}\PY{l+m+mi}{0}\PY{p}{)}
        \PY{n}{show}\PY{p}{(}\PY{p}{)}
\end{Verbatim}


    \begin{center}
    \adjustimage{max size={0.9\linewidth}{0.9\paperheight}}{Assignment3_files/Assignment3_9_0.png}
    \end{center}
    { \hspace*{\fill} \\}
    
	

	
		
    \begin{Verbatim}[commandchars=\\\{\}]
{\color{incolor}In [{\color{incolor}7}]:} \PY{n}{fig2} \PY{o}{=} \PY{n}{figure}\PY{p}{(}\PY{p}{)}
        \PY{n}{grid}\PY{p}{(}\PY{k+kc}{True}\PY{p}{)}
        \PY{n}{plot}\PY{p}{(}\PY{n}{x}\PY{p}{,}\PY{n}{cos}\PY{p}{(}\PY{n}{cos}\PY{p}{(}\PY{n}{x}\PY{p}{)}\PY{p}{)}\PY{p}{)}
        \PY{n}{plot}\PY{p}{(}\PY{n}{x}\PY{p}{,}\PY{n}{coscos}\PY{p}{(}\PY{n}{x}\PY{p}{)}\PY{p}{)}
        \PY{n}{title}\PY{p}{(}\PY{l+s+s2}{\PYZdq{}}\PY{l+s+s2}{\PYZdl{}}\PY{l+s+s2}{\PYZbs{}}\PY{l+s+s2}{cos(}\PY{l+s+s2}{\PYZbs{}}\PY{l+s+s2}{cos(x))\PYZdl{} and its periodically extended version}\PY{l+s+s2}{\PYZdq{}}\PY{p}{)}
        \PY{n}{ylabel}\PY{p}{(}\PY{l+s+sa}{r}\PY{l+s+s2}{\PYZdq{}}\PY{l+s+s2}{\PYZdl{}y\PYZdl{} (log)}\PY{l+s+s2}{\PYZdq{}}\PY{p}{,} \PY{n}{fontsize} \PY{o}{=} \PY{l+m+mi}{26}\PY{p}{)}
        \PY{n}{xlabel}\PY{p}{(}\PY{l+s+sa}{r}\PY{l+s+s2}{\PYZdq{}}\PY{l+s+s2}{\PYZdl{}x\PYZdl{} (linear)}\PY{l+s+s2}{\PYZdq{}}\PY{p}{)}
        \PY{n}{legend}\PY{p}{(}\PY{p}{[}\PY{l+s+sa}{r}\PY{l+s+s2}{\PYZdq{}}\PY{l+s+s2}{\PYZdl{}}\PY{l+s+s2}{\PYZbs{}}\PY{l+s+s2}{cos(}\PY{l+s+s2}{\PYZbs{}}\PY{l+s+s2}{cos(x))\PYZdl{}}\PY{l+s+s2}{\PYZdq{}}\PY{p}{,}\PY{l+s+sa}{r}\PY{l+s+s2}{\PYZdq{}}\PY{l+s+s2}{periodic version of \PYZdl{}}\PY{l+s+s2}{\PYZbs{}}\PY{l+s+s2}{cos(}\PY{l+s+s2}{\PYZbs{}}\PY{l+s+s2}{cos(x))\PYZdl{}}\PY{l+s+s2}{\PYZdq{}}\PY{p}{]}\PY{p}{,} \PY{n}{loc}\PY{o}{=}\PY{l+m+mi}{1}\PY{p}{)}
        \PY{n}{ylim}\PY{p}{(}\PY{l+m+mf}{0.3}\PY{p}{,}\PY{l+m+mf}{1.4}\PY{p}{)}
        \PY{n}{show}\PY{p}{(}\PY{p}{)}
\end{Verbatim}


    \begin{center}
    \adjustimage{max size={0.9\linewidth}{0.9\paperheight}}{Assignment3_files/Assignment3_10_0.png}
    \end{center}
    { \hspace*{\fill} \\}
    
	

	
		
    Clearly, we observe that \(\cos(\cos(x))\) is periodic while \(e^x\) is
not.

	

	
		
    \section{Part 2}\label{part-2}

A function to find the \(n\) odd and even fourier coefficients is
written below. It estimates the coefficients by numerically evaluating
the integrals using the \emph{quad} function.

	

	
		
    \begin{Verbatim}[commandchars=\\\{\}]
{\color{incolor}In [{\color{incolor}8}]:} \PY{k}{def} \PY{n+nf}{quad\PYZus{}fourier}\PY{p}{(}\PY{n}{f}\PY{p}{,}\PY{n}{n}\PY{p}{)}\PY{p}{:}
            \PY{l+s+sd}{\PYZdq{}\PYZdq{}\PYZdq{}}
        \PY{l+s+sd}{    Find the n even and odd fourier coefficients of f, and the DC}
        \PY{l+s+sd}{    value, using quad integration.}
        \PY{l+s+sd}{    Assumes a period from 0 to 2pi.}
        \PY{l+s+sd}{    \PYZdq{}\PYZdq{}\PYZdq{}}
            
            \PY{c+c1}{\PYZsh{} functions to integrate}
            \PY{n}{u} \PY{o}{=} \PY{k}{lambda} \PY{n}{x}\PY{p}{,}\PY{n}{k} \PY{p}{:} \PY{n}{f}\PY{p}{(}\PY{n}{x}\PY{p}{)}\PY{o}{*}\PY{n}{cos}\PY{p}{(}\PY{n}{k}\PY{o}{*}\PY{n}{x}\PY{p}{)}
            \PY{n}{v} \PY{o}{=} \PY{k}{lambda} \PY{n}{x}\PY{p}{,}\PY{n}{k} \PY{p}{:} \PY{n}{f}\PY{p}{(}\PY{n}{x}\PY{p}{)}\PY{o}{*}\PY{n}{sin}\PY{p}{(}\PY{n}{k}\PY{o}{*}\PY{n}{x}\PY{p}{)}
            
            \PY{c+c1}{\PYZsh{} DC coefficient}
            \PY{n}{a0} \PY{o}{=} \PY{l+m+mi}{1}\PY{o}{/}\PY{p}{(}\PY{l+m+mi}{2}\PY{o}{*}\PY{n}{pi}\PY{p}{)}\PY{o}{*}\PY{n}{quad}\PY{p}{(}\PY{n}{f}\PY{p}{,}\PY{l+m+mi}{0}\PY{p}{,}\PY{l+m+mi}{2}\PY{o}{*}\PY{n}{pi}\PY{p}{)}\PY{p}{[}\PY{l+m+mi}{0}\PY{p}{]}
            
            \PY{c+c1}{\PYZsh{} find coefficients by integrating}
            \PY{n}{ret} \PY{o}{=} \PY{p}{[}\PY{n}{a0}\PY{p}{]}
            \PY{k}{for} \PY{n}{k} \PY{o+ow}{in} \PY{n}{arange}\PY{p}{(}\PY{n}{n}\PY{p}{)}\PY{o}{+}\PY{l+m+mi}{1}\PY{p}{:}
                \PY{n}{ak} \PY{o}{=} \PY{l+m+mi}{1}\PY{o}{/}\PY{n}{pi}\PY{o}{*}\PY{n}{quad}\PY{p}{(}\PY{n}{u}\PY{p}{,}\PY{l+m+mi}{0}\PY{p}{,}\PY{l+m+mi}{2}\PY{o}{*}\PY{n}{pi}\PY{p}{,}\PY{n}{args}\PY{o}{=}\PY{p}{(}\PY{n}{k}\PY{p}{)}\PY{p}{)}\PY{p}{[}\PY{l+m+mi}{0}\PY{p}{]}
                \PY{n}{bk} \PY{o}{=} \PY{l+m+mi}{1}\PY{o}{/}\PY{n}{pi}\PY{o}{*}\PY{n}{quad}\PY{p}{(}\PY{n}{v}\PY{p}{,}\PY{l+m+mi}{0}\PY{p}{,}\PY{l+m+mi}{2}\PY{o}{*}\PY{n}{pi}\PY{p}{,}\PY{n}{args}\PY{o}{=}\PY{p}{(}\PY{n}{k}\PY{p}{)}\PY{p}{)}\PY{p}{[}\PY{l+m+mi}{0}\PY{p}{]}
                \PY{n}{ret}\PY{o}{.}\PY{n}{append}\PY{p}{(}\PY{n}{ak}\PY{p}{)}
                \PY{n}{ret}\PY{o}{.}\PY{n}{append}\PY{p}{(}\PY{n}{bk}\PY{p}{)}
                
            \PY{k}{return} \PY{n}{array}\PY{p}{(}\PY{n}{ret}\PY{p}{)}
\end{Verbatim}


	

	
		
    The first \(51\) coefficients for the two periodic functions are found
below:

	

	
		
    \begin{Verbatim}[commandchars=\\\{\}]
{\color{incolor}In [{\color{incolor}9}]:} \PY{n}{ef} \PY{o}{=} \PY{n}{quad\PYZus{}fourier}\PY{p}{(}\PY{n}{per\PYZus{}e}\PY{p}{,}\PY{l+m+mi}{25}\PY{p}{)}
        \PY{n}{cf} \PY{o}{=} \PY{n}{quad\PYZus{}fourier}\PY{p}{(}\PY{n}{coscos}\PY{p}{,}\PY{l+m+mi}{25}\PY{p}{)}
\end{Verbatim}


	

	
		
    \section{Part 3}\label{part-3}

The coefficients are plotted on a semilog and log-log scale below.

	

	
		
    \begin{Verbatim}[commandchars=\\\{\}]
{\color{incolor}In [{\color{incolor}10}]:} \PY{k}{def} \PY{n+nf}{plotCoeffs}\PY{p}{(}\PY{n}{coeffs}\PY{p}{,} \PY{n}{name}\PY{o}{=}\PY{l+s+s2}{\PYZdq{}}\PY{l+s+s2}{Coefficients}\PY{l+s+s2}{\PYZdq{}}\PY{p}{)}\PY{p}{:}
             \PY{l+s+sd}{\PYZdq{}\PYZdq{}\PYZdq{}}
         \PY{l+s+sd}{    Helper function to scatter the given coefficients on a }
         \PY{l+s+sd}{    semilog and loglog scale}
         \PY{l+s+sd}{    \PYZdq{}\PYZdq{}\PYZdq{}}
             
             \PY{n}{fig0} \PY{o}{=} \PY{n}{figure}\PY{p}{(}\PY{p}{)}
             \PY{n}{ax} \PY{o}{=} \PY{n}{fig0}\PY{o}{.}\PY{n}{add\PYZus{}subplot}\PY{p}{(}\PY{l+m+mi}{1}\PY{p}{,}\PY{l+m+mi}{1}\PY{p}{,}\PY{l+m+mi}{1}\PY{p}{)}
             \PY{n}{ax}\PY{o}{.}\PY{n}{set\PYZus{}yscale}\PY{p}{(}\PY{l+s+s1}{\PYZsq{}}\PY{l+s+s1}{log}\PY{l+s+s1}{\PYZsq{}}\PY{p}{)}
             \PY{n}{grid}\PY{p}{(}\PY{k+kc}{True}\PY{p}{)}
             \PY{n}{title}\PY{p}{(}\PY{l+s+s2}{\PYZdq{}}\PY{l+s+s2}{Fourier coefficients of }\PY{l+s+si}{\PYZob{}\PYZcb{}}\PY{l+s+s2}{ (semilog)}\PY{l+s+s2}{\PYZdq{}}\PY{o}{.}\PY{n}{format}\PY{p}{(}\PY{n}{name}\PY{p}{)}\PY{p}{)}
             \PY{n}{ylabel}\PY{p}{(}\PY{l+s+sa}{r}\PY{l+s+s2}{\PYZdq{}}\PY{l+s+s2}{Magnitude of coefficients}\PY{l+s+s2}{\PYZdq{}}\PY{p}{,} \PY{n}{fontsize} \PY{o}{=} \PY{l+m+mi}{26}\PY{p}{)}
             \PY{n}{xlabel}\PY{p}{(}\PY{l+s+sa}{r}\PY{l+s+s2}{\PYZdq{}}\PY{l+s+s2}{Index}\PY{l+s+s2}{\PYZdq{}}\PY{p}{)}
             \PY{n}{scatter}\PY{p}{(}\PY{n}{arange}\PY{p}{(}\PY{n+nb}{len}\PY{p}{(}\PY{n}{coeffs}\PY{p}{)}\PY{p}{)}\PY{o}{+}\PY{l+m+mi}{1}\PY{p}{,}\PY{n+nb}{abs}\PY{p}{(}\PY{n}{coeffs}\PY{p}{)}\PY{p}{,}\PY{n}{color}\PY{o}{=}\PY{l+s+s1}{\PYZsq{}}\PY{l+s+s1}{red}\PY{l+s+s1}{\PYZsq{}}\PY{p}{,}\PY{n}{s}\PY{o}{=}\PY{l+m+mi}{50}\PY{p}{)}
             \PY{n}{show}\PY{p}{(}\PY{p}{)}
             
             \PY{n}{fig1} \PY{o}{=} \PY{n}{figure}\PY{p}{(}\PY{p}{)}
             \PY{n}{ax} \PY{o}{=} \PY{n}{fig1}\PY{o}{.}\PY{n}{add\PYZus{}subplot}\PY{p}{(}\PY{l+m+mi}{1}\PY{p}{,}\PY{l+m+mi}{1}\PY{p}{,}\PY{l+m+mi}{1}\PY{p}{)}
             \PY{n}{ax}\PY{o}{.}\PY{n}{set\PYZus{}yscale}\PY{p}{(}\PY{l+s+s1}{\PYZsq{}}\PY{l+s+s1}{log}\PY{l+s+s1}{\PYZsq{}}\PY{p}{)}
             \PY{n}{ax}\PY{o}{.}\PY{n}{set\PYZus{}xscale}\PY{p}{(}\PY{l+s+s1}{\PYZsq{}}\PY{l+s+s1}{log}\PY{l+s+s1}{\PYZsq{}}\PY{p}{)}
             \PY{n}{grid}\PY{p}{(}\PY{k+kc}{True}\PY{p}{)}
             \PY{n}{title}\PY{p}{(}\PY{l+s+s2}{\PYZdq{}}\PY{l+s+s2}{Fourier coefficients of }\PY{l+s+si}{\PYZob{}\PYZcb{}}\PY{l+s+s2}{ (log\PYZhy{}log)}\PY{l+s+s2}{\PYZdq{}}\PY{o}{.}\PY{n}{format}\PY{p}{(}\PY{n}{name}\PY{p}{)}\PY{p}{)}
             \PY{n}{ylabel}\PY{p}{(}\PY{l+s+sa}{r}\PY{l+s+s2}{\PYZdq{}}\PY{l+s+s2}{Magnitude of coefficients}\PY{l+s+s2}{\PYZdq{}}\PY{p}{,} \PY{n}{fontsize} \PY{o}{=} \PY{l+m+mi}{26}\PY{p}{)}
             \PY{n}{xlabel}\PY{p}{(}\PY{l+s+sa}{r}\PY{l+s+s2}{\PYZdq{}}\PY{l+s+s2}{Index}\PY{l+s+s2}{\PYZdq{}}\PY{p}{)}
             \PY{n}{scatter}\PY{p}{(}\PY{n}{arange}\PY{p}{(}\PY{n+nb}{len}\PY{p}{(}\PY{n}{coeffs}\PY{p}{)}\PY{p}{)}\PY{o}{+}\PY{l+m+mi}{1}\PY{p}{,}\PY{n+nb}{abs}\PY{p}{(}\PY{n}{coeffs}\PY{p}{)}\PY{p}{,}\PY{n}{color}\PY{o}{=}\PY{l+s+s1}{\PYZsq{}}\PY{l+s+s1}{red}\PY{l+s+s1}{\PYZsq{}}\PY{p}{,}\PY{n}{s}\PY{o}{=}\PY{l+m+mi}{50}\PY{p}{)}
             
             \PY{k}{return} \PY{n}{fig0}\PY{p}{,}\PY{n}{fig1}
\end{Verbatim}


	

	
		
    \begin{Verbatim}[commandchars=\\\{\}]
{\color{incolor}In [{\color{incolor}11}]:} \PY{n}{fig3}\PY{p}{,}\PY{n}{fig4} \PY{o}{=} \PY{n}{plotCoeffs}\PY{p}{(}\PY{n}{ef}\PY{p}{,}\PY{l+s+sa}{r}\PY{l+s+s2}{\PYZdq{}}\PY{l+s+s2}{\PYZdl{}e\PYZca{}x\PYZdl{}}\PY{l+s+s2}{\PYZdq{}}\PY{p}{)}
\end{Verbatim}


    \begin{center}
    \adjustimage{max size={0.9\linewidth}{0.9\paperheight}}{Assignment3_files/Assignment3_18_0.png}
    \end{center}
    { \hspace*{\fill} \\}
    
    \begin{center}
    \adjustimage{max size={0.9\linewidth}{0.9\paperheight}}{Assignment3_files/Assignment3_18_1.png}
    \end{center}
    { \hspace*{\fill} \\}
    
	

	
		
    \begin{Verbatim}[commandchars=\\\{\}]
{\color{incolor}In [{\color{incolor}12}]:} \PY{n}{fig5}\PY{p}{,}\PY{n}{fig6} \PY{o}{=} \PY{n}{plotCoeffs}\PY{p}{(}\PY{n}{cf}\PY{p}{,}\PY{l+s+s2}{\PYZdq{}}\PY{l+s+s2}{\PYZdl{}}\PY{l+s+s2}{\PYZbs{}}\PY{l+s+s2}{cos(}\PY{l+s+s2}{\PYZbs{}}\PY{l+s+s2}{cos(x))\PYZdl{}}\PY{l+s+s2}{\PYZdq{}}\PY{p}{)}
\end{Verbatim}


    \begin{center}
    \adjustimage{max size={0.9\linewidth}{0.9\paperheight}}{Assignment3_files/Assignment3_19_0.png}
    \end{center}
    { \hspace*{\fill} \\}
    
    \begin{center}
    \adjustimage{max size={0.9\linewidth}{0.9\paperheight}}{Assignment3_files/Assignment3_19_1.png}
    \end{center}
    { \hspace*{\fill} \\}
    
	

	
		
    \subsection{Question (a)}\label{question-a}

The \(b_n\) coefficients correspond to the coefficients for the
\(sin(kx)\) harmonics, which are odd functions. Since \(\cos(\cos(x))\)
is an even function, it does not contain any odd harmonics. So, the
\(b_n\) coefficients are very nearly zero in Fig 3 and 4.

	

	
		
    \subsection{Question (b)}\label{question-b}

The magnitude of the fourier coefficients corresponds to the amount of a
particular frequency which is present in the function. Heuristically, we
can observe that \(\cos(\cos(x))\) is a function which contains not many
different frequencies of harmonics, so one may expect that high
frequency harmonics will not contribute to the fourier series much. On
the other hand, it is not evident or obvious what the frequency
components of \(e^x\) are. The reason for the slow decay of the
coefficients of \(e^x\) is the presence of a discontinuity in its
periodic extension. Clearly, the discontinuity occurs at all
\(x = 2k\pi\) where \(k\) is any integer. To accurately represent
discontiuities using a series of continuous harmonic functions requires
non-uniform convergence. In other words, the discontinuity represents a
"high frequency" component in the function. So, the coefficients do not
decay as fast.

	

	
		
    \subsection{Question (c)}\label{question-c}

The \emph{loglog} plot looks linear for the first case because the
fourier coefficients of \(e^x\) decay as a power of \(n\), in
particular, \(a_n\) decays as \(\frac{1}{n}\) and \(b_n\) decays as
\(\frac{1}{n^2}\). The magnitude of the complex fourier coefficients of
\(e^x\) therefore decays as \(\frac{1}{n}\). This is because of the
discontinuities in the periodic extension of \(e^x\). The \emph{semilog}
plot looks linear for the second case because the fourier coefficients
of \(\cos(\cos(x))\) decay exponentially with \(n\).

	

	
		
    \section{Part 4}\label{part-4}

The fourier coefficients are now estimated using a different approach
than numerical integration - least squares estimation. A function to do
the same is written below:

	

	
		
    \begin{Verbatim}[commandchars=\\\{\}]
{\color{incolor}In [{\color{incolor}13}]:} \PY{k}{def} \PY{n+nf}{lstsq\PYZus{}fourier}\PY{p}{(}\PY{n}{f}\PY{p}{,}\PY{n}{n}\PY{p}{,}\PY{n}{steps}\PY{o}{=}\PY{l+m+mi}{400}\PY{p}{)}\PY{p}{:}
             \PY{l+s+sd}{\PYZdq{}\PYZdq{}\PYZdq{}}
         \PY{l+s+sd}{    Find the n even and odd fourier coefficients of f, and the DC}
         \PY{l+s+sd}{    value, using least squares estimation. Defaults to 400 steps.}
         \PY{l+s+sd}{    Assumes a period from 0 to 2pi.}
         \PY{l+s+sd}{    }
         \PY{l+s+sd}{    Also returns the least squares matrix}
         \PY{l+s+sd}{    \PYZdq{}\PYZdq{}\PYZdq{}}
             
             \PY{n}{x}\PY{o}{=}\PY{n}{linspace}\PY{p}{(}\PY{l+m+mi}{0}\PY{p}{,}\PY{l+m+mi}{2}\PY{o}{*}\PY{n}{pi}\PY{p}{,}\PY{n}{steps}\PY{o}{+}\PY{l+m+mi}{1}\PY{p}{)}
             \PY{n}{x}\PY{o}{=}\PY{n}{x}\PY{p}{[}\PY{p}{:}\PY{o}{\PYZhy{}}\PY{l+m+mi}{1}\PY{p}{]} \PY{c+c1}{\PYZsh{} drop last term to have a proper periodic integral}
             \PY{n}{b}\PY{o}{=}\PY{n}{f}\PY{p}{(}\PY{n}{x}\PY{p}{)} \PY{c+c1}{\PYZsh{} f has been written to take a vector}
             \PY{n}{A}\PY{o}{=}\PY{n}{zeros}\PY{p}{(}\PY{p}{(}\PY{n}{steps}\PY{p}{,}\PY{l+m+mi}{2}\PY{o}{*}\PY{n}{n}\PY{o}{+}\PY{l+m+mi}{1}\PY{p}{)}\PY{p}{)} \PY{c+c1}{\PYZsh{} allocate space for A}
             \PY{n}{A}\PY{p}{[}\PY{p}{:}\PY{p}{,}\PY{l+m+mi}{0}\PY{p}{]}\PY{o}{=}\PY{l+m+mi}{1} \PY{c+c1}{\PYZsh{} col 1 is all ones}
             \PY{k}{for} \PY{n}{k} \PY{o+ow}{in} \PY{n+nb}{range}\PY{p}{(}\PY{l+m+mi}{1}\PY{p}{,}\PY{n}{n}\PY{o}{+}\PY{l+m+mi}{1}\PY{p}{)}\PY{p}{:}
                 \PY{n}{A}\PY{p}{[}\PY{p}{:}\PY{p}{,}\PY{l+m+mi}{2}\PY{o}{*}\PY{n}{k}\PY{o}{\PYZhy{}}\PY{l+m+mi}{1}\PY{p}{]}\PY{o}{=}\PY{n}{cos}\PY{p}{(}\PY{n}{k}\PY{o}{*}\PY{n}{x}\PY{p}{)} \PY{c+c1}{\PYZsh{} cos(kx) column}
                 \PY{n}{A}\PY{p}{[}\PY{p}{:}\PY{p}{,}\PY{l+m+mi}{2}\PY{o}{*}\PY{n}{k}\PY{p}{]}\PY{o}{=}\PY{n}{sin}\PY{p}{(}\PY{n}{k}\PY{o}{*}\PY{n}{x}\PY{p}{)} \PY{c+c1}{\PYZsh{} sin(kx) column}
             
             \PY{k}{return} \PY{n}{lstsq}\PY{p}{(}\PY{n}{A}\PY{p}{,}\PY{n}{b}\PY{p}{)}\PY{p}{[}\PY{l+m+mi}{0}\PY{p}{]}\PY{p}{,} \PY{n}{A}
\end{Verbatim}


	

	
		
    \section{Part 5}\label{part-5}

The coefficients obtained using this method are plotted on the same
figures below:

	

	
		
    \begin{Verbatim}[commandchars=\\\{\}]
{\color{incolor}In [{\color{incolor}14}]:} \PY{n}{s} \PY{o}{=} \PY{l+m+mi}{400} \PY{c+c1}{\PYZsh{} number of steps for least square estimation}
         \PY{n}{elc}\PY{p}{,} \PY{n}{Aexp} \PY{o}{=} \PY{n}{lstsq\PYZus{}fourier}\PY{p}{(}\PY{n}{per\PYZus{}e}\PY{p}{,}\PY{l+m+mi}{25}\PY{p}{,}\PY{n}{s}\PY{p}{)}
         \PY{n}{ax} \PY{o}{=} \PY{n}{fig3}\PY{o}{.}\PY{n}{axes}\PY{p}{[}\PY{l+m+mi}{0}\PY{p}{]}
         \PY{n}{ax}\PY{o}{.}\PY{n}{scatter}\PY{p}{(}\PY{n}{arange}\PY{p}{(}\PY{n+nb}{len}\PY{p}{(}\PY{n}{elc}\PY{p}{)}\PY{p}{)}\PY{o}{+}\PY{l+m+mi}{1}\PY{p}{,}\PY{n+nb}{abs}\PY{p}{(}\PY{n}{elc}\PY{p}{)}\PY{p}{,}\PY{n}{s}\PY{o}{=}\PY{l+m+mi}{50}\PY{p}{,}\PY{n}{color}\PY{o}{=}\PY{l+s+s1}{\PYZsq{}}\PY{l+s+s1}{green}\PY{l+s+s1}{\PYZsq{}}\PY{p}{)}
         \PY{n}{ax}\PY{o}{.}\PY{n}{legend}\PY{p}{(}\PY{p}{[}\PY{l+s+s2}{\PYZdq{}}\PY{l+s+s2}{Using quad integration}\PY{l+s+s2}{\PYZdq{}}\PY{p}{,} \PY{l+s+s2}{\PYZdq{}}\PY{l+s+s2}{Using least squares}\PY{l+s+s2}{\PYZdq{}}\PY{p}{]}\PY{p}{,} \PY{n}{loc}\PY{o}{=}\PY{l+m+mi}{3}\PY{p}{)}
         \PY{n}{fig3}
\end{Verbatim}

\texttt{\color{outcolor}Out[{\color{outcolor}14}]:}
    
    \begin{center}
    \adjustimage{max size={0.9\linewidth}{0.9\paperheight}}{Assignment3_files/Assignment3_26_0.png}
    \end{center}
    { \hspace*{\fill} \\}
    

	

	
		
    \begin{Verbatim}[commandchars=\\\{\}]
{\color{incolor}In [{\color{incolor}15}]:} \PY{n}{ax} \PY{o}{=} \PY{n}{fig4}\PY{o}{.}\PY{n}{axes}\PY{p}{[}\PY{l+m+mi}{0}\PY{p}{]}
         \PY{n}{ax}\PY{o}{.}\PY{n}{scatter}\PY{p}{(}\PY{n}{arange}\PY{p}{(}\PY{n+nb}{len}\PY{p}{(}\PY{n}{elc}\PY{p}{)}\PY{p}{)}\PY{o}{+}\PY{l+m+mi}{1}\PY{p}{,}\PY{n+nb}{abs}\PY{p}{(}\PY{n}{elc}\PY{p}{)}\PY{p}{,}\PY{n}{color}\PY{o}{=}\PY{l+s+s1}{\PYZsq{}}\PY{l+s+s1}{green}\PY{l+s+s1}{\PYZsq{}}\PY{p}{,}\PY{n}{s}\PY{o}{=}\PY{l+m+mi}{50}\PY{p}{)}
         \PY{n}{ax}\PY{o}{.}\PY{n}{legend}\PY{p}{(}\PY{p}{[}\PY{l+s+s2}{\PYZdq{}}\PY{l+s+s2}{Using quad integration}\PY{l+s+s2}{\PYZdq{}}\PY{p}{,} \PY{l+s+s2}{\PYZdq{}}\PY{l+s+s2}{Using least squares}\PY{l+s+s2}{\PYZdq{}}\PY{p}{]}\PY{p}{,}\PY{n}{loc}\PY{o}{=}\PY{l+m+mi}{3}\PY{p}{)}
         \PY{n}{fig4}
\end{Verbatim}

\texttt{\color{outcolor}Out[{\color{outcolor}15}]:}
    
    \begin{center}
    \adjustimage{max size={0.9\linewidth}{0.9\paperheight}}{Assignment3_files/Assignment3_27_0.png}
    \end{center}
    { \hspace*{\fill} \\}
    

	

	
		
    \begin{Verbatim}[commandchars=\\\{\}]
{\color{incolor}In [{\color{incolor}16}]:} \PY{n}{clc}\PY{p}{,} \PY{n}{Acos} \PY{o}{=} \PY{n}{lstsq\PYZus{}fourier}\PY{p}{(}\PY{n}{coscos}\PY{p}{,}\PY{l+m+mi}{25}\PY{p}{,}\PY{n}{s}\PY{p}{)}
         \PY{n}{ax} \PY{o}{=} \PY{n}{fig5}\PY{o}{.}\PY{n}{axes}\PY{p}{[}\PY{l+m+mi}{0}\PY{p}{]}
         \PY{n}{ax}\PY{o}{.}\PY{n}{scatter}\PY{p}{(}\PY{n}{arange}\PY{p}{(}\PY{n+nb}{len}\PY{p}{(}\PY{n}{clc}\PY{p}{)}\PY{p}{)}\PY{o}{+}\PY{l+m+mi}{1}\PY{p}{,}\PY{n+nb}{abs}\PY{p}{(}\PY{n}{clc}\PY{p}{)}\PY{p}{,}\PY{n}{color}\PY{o}{=}\PY{l+s+s1}{\PYZsq{}}\PY{l+s+s1}{green}\PY{l+s+s1}{\PYZsq{}}\PY{p}{,}\PY{n}{s}\PY{o}{=}\PY{l+m+mi}{50}\PY{p}{)}
         \PY{n}{ax}\PY{o}{.}\PY{n}{legend}\PY{p}{(}\PY{p}{[}\PY{l+s+s2}{\PYZdq{}}\PY{l+s+s2}{Using quad integration}\PY{l+s+s2}{\PYZdq{}}\PY{p}{,} \PY{l+s+s2}{\PYZdq{}}\PY{l+s+s2}{Using least squares}\PY{l+s+s2}{\PYZdq{}}\PY{p}{]}\PY{p}{,}\PY{n}{loc}\PY{o}{=}\PY{l+m+mi}{1}\PY{p}{)}
         \PY{n}{fig5}
\end{Verbatim}

\texttt{\color{outcolor}Out[{\color{outcolor}16}]:}
    
    \begin{center}
    \adjustimage{max size={0.9\linewidth}{0.9\paperheight}}{Assignment3_files/Assignment3_28_0.png}
    \end{center}
    { \hspace*{\fill} \\}
    

	

	
		
    \begin{Verbatim}[commandchars=\\\{\}]
{\color{incolor}In [{\color{incolor}17}]:} \PY{n}{ax} \PY{o}{=} \PY{n}{fig6}\PY{o}{.}\PY{n}{axes}\PY{p}{[}\PY{l+m+mi}{0}\PY{p}{]}
         \PY{n}{ax}\PY{o}{.}\PY{n}{scatter}\PY{p}{(}\PY{n}{arange}\PY{p}{(}\PY{n+nb}{len}\PY{p}{(}\PY{n}{clc}\PY{p}{)}\PY{p}{)}\PY{o}{+}\PY{l+m+mi}{1}\PY{p}{,}\PY{n+nb}{abs}\PY{p}{(}\PY{n}{clc}\PY{p}{)}\PY{p}{,}\PY{n}{color}\PY{o}{=}\PY{l+s+s1}{\PYZsq{}}\PY{l+s+s1}{green}\PY{l+s+s1}{\PYZsq{}}\PY{p}{,}\PY{n}{s}\PY{o}{=}\PY{l+m+mi}{50}\PY{p}{)}
         \PY{n}{ax}\PY{o}{.}\PY{n}{legend}\PY{p}{(}\PY{p}{[}\PY{l+s+s2}{\PYZdq{}}\PY{l+s+s2}{Using quad integration}\PY{l+s+s2}{\PYZdq{}}\PY{p}{,} \PY{l+s+s2}{\PYZdq{}}\PY{l+s+s2}{Using least squares}\PY{l+s+s2}{\PYZdq{}}\PY{p}{]}\PY{p}{,}\PY{n}{loc}\PY{o}{=}\PY{l+m+mi}{6}\PY{p}{)}
         \PY{n}{fig6}
\end{Verbatim}

\texttt{\color{outcolor}Out[{\color{outcolor}17}]:}
    
    \begin{center}
    \adjustimage{max size={0.9\linewidth}{0.9\paperheight}}{Assignment3_files/Assignment3_29_0.png}
    \end{center}
    { \hspace*{\fill} \\}
    

	

	
		
    \section{Part 6}\label{part-6}

From the above plots we observe that the coefficients more or less agree
for the second case, namely \(\cos(\cos(x))\), but disagree
significantly in the first case, namely \(e^x\). This is to be expected
because the numerical integration method acquires more information about
the function than merely the value at just \(400\) points, like the
least squares estimation does. Since \(e^x\) contains more frequency
components, more samples of the function are required to accurately
estimate its fourier coefficients. The deviations are obtained below:

	

	
		
    \begin{Verbatim}[commandchars=\\\{\}]
{\color{incolor}In [{\color{incolor}18}]:} \PY{n+nb}{print}\PY{p}{(}\PY{n+nb}{max}\PY{p}{(}\PY{n+nb}{abs}\PY{p}{(}\PY{n}{ef}\PY{o}{\PYZhy{}}\PY{n}{elc}\PY{p}{)}\PY{p}{)}\PY{p}{)} \PY{c+c1}{\PYZsh{} largest deviation in coefficients of e\PYZca{}x}
\end{Verbatim}


    \begin{Verbatim}[commandchars=\\\{\}]
1.33273087034

    \end{Verbatim}

	

	
		
    \begin{Verbatim}[commandchars=\\\{\}]
{\color{incolor}In [{\color{incolor}19}]:} \PY{n+nb}{print}\PY{p}{(}\PY{n+nb}{max}\PY{p}{(}\PY{n+nb}{abs}\PY{p}{(}\PY{n}{cf}\PY{o}{\PYZhy{}}\PY{n}{clc}\PY{p}{)}\PY{p}{)}\PY{p}{)} \PY{c+c1}{\PYZsh{} largest deviation in coefficients of cos(cos(x))}
\end{Verbatim}


    \begin{Verbatim}[commandchars=\\\{\}]
2.57586713575e-15

    \end{Verbatim}

	

	
		
    To obtain better results for \(e^x\), one can perform the least squares
technique using more than just \(400\) points. We compare the
coefficients obtained using \(10^5\) steps with the previous results
below:

	

	
		
    \begin{Verbatim}[commandchars=\\\{\}]
{\color{incolor}In [{\color{incolor}20}]:} \PY{n}{el\PYZus{}better}\PY{p}{,} \PY{n}{Aexp\PYZus{}better} \PY{o}{=} \PY{n}{lstsq\PYZus{}fourier}\PY{p}{(}\PY{n}{per\PYZus{}e}\PY{p}{,}\PY{l+m+mi}{25}\PY{p}{,}\PY{n+nb}{int}\PY{p}{(}\PY{l+m+mf}{1e5}\PY{p}{)}\PY{p}{)}
\end{Verbatim}


	

	
		
    \begin{Verbatim}[commandchars=\\\{\}]
{\color{incolor}In [{\color{incolor}21}]:} \PY{n}{ax} \PY{o}{=} \PY{n}{fig4}\PY{o}{.}\PY{n}{axes}\PY{p}{[}\PY{l+m+mi}{0}\PY{p}{]}
         \PY{n}{ax}\PY{o}{.}\PY{n}{scatter}\PY{p}{(}\PY{n}{arange}\PY{p}{(}\PY{n+nb}{len}\PY{p}{(}\PY{n}{el\PYZus{}better}\PY{p}{)}\PY{p}{)}\PY{o}{+}\PY{l+m+mi}{1}\PY{p}{,}\PY{n+nb}{abs}\PY{p}{(}\PY{n}{el\PYZus{}better}\PY{p}{)}\PY{p}{,}\PY{n}{color}\PY{o}{=}\PY{l+s+s1}{\PYZsq{}}\PY{l+s+s1}{blue}\PY{l+s+s1}{\PYZsq{}}\PY{p}{,}\PY{n}{s}\PY{o}{=}\PY{l+m+mi}{30}\PY{p}{)}
         \PY{n}{ax}\PY{o}{.}\PY{n}{legend}\PY{p}{(}\PY{p}{[}\PY{l+s+s2}{\PYZdq{}}\PY{l+s+s2}{Using quad integration}\PY{l+s+s2}{\PYZdq{}}\PY{p}{,} 
                    \PY{l+s+s2}{\PYZdq{}}\PY{l+s+s2}{Using least squares(400 steps)}\PY{l+s+s2}{\PYZdq{}}\PY{p}{,}\PY{l+s+sa}{r}\PY{l+s+s2}{\PYZdq{}}\PY{l+s+s2}{Using least squares(\PYZdl{}10\PYZca{}5\PYZdl{} steps)}\PY{l+s+s2}{\PYZdq{}}\PY{p}{]}
                   \PY{p}{,}\PY{n}{loc}\PY{o}{=}\PY{l+m+mi}{3}\PY{p}{)}
         \PY{n}{fig4}
\end{Verbatim}

\texttt{\color{outcolor}Out[{\color{outcolor}21}]:}
    
    \begin{center}
    \adjustimage{max size={0.9\linewidth}{0.9\paperheight}}{Assignment3_files/Assignment3_35_0.png}
    \end{center}
    { \hspace*{\fill} \\}
    

	

	
		
    We observe that the coefficients are now in much more agreement. This
can be seen by calculating the maximum devation again:

	

	
		
    \begin{Verbatim}[commandchars=\\\{\}]
{\color{incolor}In [{\color{incolor}22}]:} \PY{c+c1}{\PYZsh{} largest deviation in coefficients of e\PYZca{}x using 1e5 steps}
         \PY{n+nb}{print}\PY{p}{(}\PY{n+nb}{max}\PY{p}{(}\PY{n+nb}{abs}\PY{p}{(}\PY{n}{ef}\PY{o}{\PYZhy{}}\PY{n}{el\PYZus{}better}\PY{p}{)}\PY{p}{)}\PY{p}{)} 
\end{Verbatim}


    \begin{Verbatim}[commandchars=\\\{\}]
0.00534486058358

    \end{Verbatim}

	

	
		
    The above behaviour can be explained by comparing how the fourier
coefficients are computed by quad and least squares. Both methods
attempt to solve the matrix problem

\[Ac=b\]

Where \(A\) is the matrix of sampled harmonics, \(c\) is the vector of
coefficients and \(b\) is the vector of sampled function values.

When the coefficients are computed using the integration formulas, what
essentially occurs is that the matrix \(A\) is assumed to be orthogonal.
This means that \(A^T A\) is a diagonal matrix, and with appropriate
normalization, is the identity matrix. This means that the matrix
equation can be turned into:

\[A^T A c = A^T b\] \[c = K A^T b\]

The vector obtained by computing \(A^T b\) after normalization with the
constant \(K\) is the vector of coefficients. If we observe carefully,
this dot product is analogous to performing trapezoidal integration in
accordance with the integration formulas for the fourier coefficients.
The quad function performs a more accurate version of the integration by
using interpolation, but the essence of the procedure remains the same.

The least squares approach computes the coefficients without presuming
anything about the matrix \(A\). It finds \(c\) as follows:

\[c = (A^T A)^{-1} A^T b\]

Since the matrix \(A\) becomes more orthogonal as the number of samples
of the harmonics increases, we can conclude that the two results
converge to the same coefficients as the number of samples increases.

The deviation in the coefficients of \(e^x\) which we observed with
\(400\) points can be explained by noticing that \(e^x\) has more high
frequency components than \(\cos(\cos(x))\), which means that the fact
that \(A\) is not exactly orthogonal will have a greater affect. Also,
the quad function does not take the number of samples as an argument,
which means that we cannot be sure that the two methods are on an equal
footing when taking \(400\) samples for least squares.

	

	
		
    \section{Part 7}\label{part-7}

The function values are estimated using the evaluated fourier
coefficients below:

	

	
		
    \begin{Verbatim}[commandchars=\\\{\}]
{\color{incolor}In [{\color{incolor}23}]:} \PY{n}{I} \PY{o}{=} \PY{n}{linspace}\PY{p}{(}\PY{l+m+mi}{0}\PY{p}{,}\PY{l+m+mi}{2}\PY{o}{*}\PY{n}{pi}\PY{p}{,}\PY{n}{s}\PY{p}{)} \PY{c+c1}{\PYZsh{} x values for the evaluated function values}
\end{Verbatim}


	

	
		
    \begin{Verbatim}[commandchars=\\\{\}]
{\color{incolor}In [{\color{incolor}24}]:} \PY{n}{ax} \PY{o}{=} \PY{n}{fig1}\PY{o}{.}\PY{n}{axes}\PY{p}{[}\PY{l+m+mi}{0}\PY{p}{]}
         \PY{n}{ax}\PY{o}{.}\PY{n}{scatter}\PY{p}{(}\PY{n}{I}\PY{p}{,}\PY{n}{dot}\PY{p}{(}\PY{n}{Aexp}\PY{p}{,}\PY{n}{ef}\PY{p}{)}\PY{p}{,}\PY{n}{color} \PY{o}{=} \PY{l+s+s1}{\PYZsq{}}\PY{l+s+s1}{green}\PY{l+s+s1}{\PYZsq{}}\PY{p}{)}
         \PY{n}{ax}\PY{o}{.}\PY{n}{scatter}\PY{p}{(}\PY{n}{I}\PY{p}{,}\PY{n}{dot}\PY{p}{(}\PY{n}{Aexp}\PY{p}{,}\PY{n}{elc}\PY{p}{)}\PY{p}{,}\PY{n}{color} \PY{o}{=} \PY{l+s+s1}{\PYZsq{}}\PY{l+s+s1}{black}\PY{l+s+s1}{\PYZsq{}}\PY{p}{,}\PY{n}{marker}\PY{o}{=}\PY{l+s+s1}{\PYZsq{}}\PY{l+s+s1}{+}\PY{l+s+s1}{\PYZsq{}}\PY{p}{)}
         \PY{n}{ax}\PY{o}{.}\PY{n}{set\PYZus{}ylim}\PY{p}{(}\PY{l+m+mf}{1e\PYZhy{}3}\PY{p}{,}\PY{l+m+mf}{1e6}\PY{p}{)}
         \PY{n}{ax}\PY{o}{.}\PY{n}{legend}\PY{p}{(}\PY{p}{[}\PY{l+s+sa}{r}\PY{l+s+s2}{\PYZdq{}}\PY{l+s+s2}{\PYZdl{}e\PYZca{}x\PYZdl{}}\PY{l+s+s2}{\PYZdq{}}\PY{p}{,}\PY{l+s+sa}{r}\PY{l+s+s2}{\PYZdq{}}\PY{l+s+s2}{periodic version of \PYZdl{}e\PYZca{}x\PYZdl{}}\PY{l+s+s2}{\PYZdq{}}\PY{p}{,}
                    \PY{l+s+s2}{\PYZdq{}}\PY{l+s+s2}{quad estimated}\PY{l+s+s2}{\PYZdq{}}\PY{p}{,}\PY{l+s+s2}{\PYZdq{}}\PY{l+s+s2}{Least squares estimated}\PY{l+s+s2}{\PYZdq{}}\PY{p}{]}\PY{p}{,} \PY{n}{loc} \PY{o}{=} \PY{l+m+mi}{4}\PY{p}{)}
         \PY{n}{fig1}
\end{Verbatim}

\texttt{\color{outcolor}Out[{\color{outcolor}24}]:}
    
    \begin{center}
    \adjustimage{max size={0.9\linewidth}{0.9\paperheight}}{Assignment3_files/Assignment3_41_0.png}
    \end{center}
    { \hspace*{\fill} \\}
    

	

	
		
    We can gain more insight by plotting the estimate as a line plot on a
linear scale.

	

	
		
    \begin{Verbatim}[commandchars=\\\{\}]
{\color{incolor}In [{\color{incolor}25}]:} \PY{n}{figure}\PY{p}{(}\PY{p}{)}
         \PY{n}{grid}\PY{p}{(}\PY{k+kc}{True}\PY{p}{)}
         \PY{n}{xlim}\PY{p}{(}\PY{o}{\PYZhy{}}\PY{l+m+mi}{2}\PY{p}{,}\PY{l+m+mi}{7}\PY{p}{)}
         \PY{n}{plot}\PY{p}{(}\PY{n}{x}\PY{p}{,}\PY{n}{per\PYZus{}e}\PY{p}{(}\PY{n}{x}\PY{p}{)}\PY{p}{)}
         \PY{n}{plot}\PY{p}{(}\PY{n}{I}\PY{p}{,}\PY{n}{dot}\PY{p}{(}\PY{n}{Aexp}\PY{p}{,}\PY{n}{ef}\PY{p}{)}\PY{p}{,}\PY{n}{linewidth}\PY{o}{=}\PY{l+m+mi}{3}\PY{p}{)}
         \PY{n}{title}\PY{p}{(}\PY{l+s+s2}{\PYZdq{}}\PY{l+s+s2}{\PYZdl{}e\PYZca{}x\PYZdl{} and its fourier approximation}\PY{l+s+s2}{\PYZdq{}}\PY{p}{)}
         \PY{n}{ylabel}\PY{p}{(}\PY{l+s+sa}{r}\PY{l+s+s2}{\PYZdq{}}\PY{l+s+s2}{\PYZdl{}y\PYZdl{} (linear)}\PY{l+s+s2}{\PYZdq{}}\PY{p}{,} \PY{n}{fontsize} \PY{o}{=} \PY{l+m+mi}{26}\PY{p}{)}
         \PY{n}{xlabel}\PY{p}{(}\PY{l+s+sa}{r}\PY{l+s+s2}{\PYZdq{}}\PY{l+s+s2}{\PYZdl{}x\PYZdl{} (linear)}\PY{l+s+s2}{\PYZdq{}}\PY{p}{)}
         \PY{n}{legend}\PY{p}{(}\PY{p}{[}\PY{l+s+sa}{r}\PY{l+s+s2}{\PYZdq{}}\PY{l+s+s2}{Periodic version of \PYZdl{}e\PYZca{}x\PYZdl{}}\PY{l+s+s2}{\PYZdq{}}\PY{p}{,}\PY{l+s+sa}{r}\PY{l+s+s2}{\PYZdq{}}\PY{l+s+s2}{Fourier estimate of \PYZdl{}e\PYZca{}x\PYZdl{}}\PY{l+s+s2}{\PYZdq{}}\PY{p}{]}\PY{p}{,} \PY{n}{loc}\PY{o}{=}\PY{l+m+mi}{0}\PY{p}{)}
         \PY{n}{show}\PY{p}{(}\PY{p}{)}
\end{Verbatim}


    \begin{center}
    \adjustimage{max size={0.9\linewidth}{0.9\paperheight}}{Assignment3_files/Assignment3_43_0.png}
    \end{center}
    { \hspace*{\fill} \\}
    
	

	
		
    The above graph makes the \textbf{Gibb's phenomenon} apparent.

	

	
		
    \begin{Verbatim}[commandchars=\\\{\}]
{\color{incolor}In [{\color{incolor}26}]:} \PY{n}{ax} \PY{o}{=} \PY{n}{fig2}\PY{o}{.}\PY{n}{axes}\PY{p}{[}\PY{l+m+mi}{0}\PY{p}{]}
         \PY{n}{ax}\PY{o}{.}\PY{n}{scatter}\PY{p}{(}\PY{n}{I}\PY{p}{,}\PY{n}{dot}\PY{p}{(}\PY{n}{Acos}\PY{p}{,}\PY{n}{cf}\PY{p}{)}\PY{p}{,}\PY{n}{color} \PY{o}{=} \PY{l+s+s1}{\PYZsq{}}\PY{l+s+s1}{green}\PY{l+s+s1}{\PYZsq{}}\PY{p}{)}
         \PY{n}{ax}\PY{o}{.}\PY{n}{scatter}\PY{p}{(}\PY{n}{I}\PY{p}{,}\PY{n}{dot}\PY{p}{(}\PY{n}{Acos}\PY{p}{,}\PY{n}{clc}\PY{p}{)}\PY{p}{,}\PY{n}{color} \PY{o}{=} \PY{l+s+s1}{\PYZsq{}}\PY{l+s+s1}{black}\PY{l+s+s1}{\PYZsq{}}\PY{p}{)}
         \PY{n}{ax}\PY{o}{.}\PY{n}{legend}\PY{p}{(}\PY{p}{[}\PY{l+s+sa}{r}\PY{l+s+s2}{\PYZdq{}}\PY{l+s+s2}{\PYZdl{}}\PY{l+s+s2}{\PYZbs{}}\PY{l+s+s2}{cos(}\PY{l+s+s2}{\PYZbs{}}\PY{l+s+s2}{cos(x))\PYZdl{}}\PY{l+s+s2}{\PYZdq{}}\PY{p}{,}\PY{l+s+sa}{r}\PY{l+s+s2}{\PYZdq{}}\PY{l+s+s2}{periodic version of \PYZdl{}}\PY{l+s+s2}{\PYZbs{}}\PY{l+s+s2}{cos(}\PY{l+s+s2}{\PYZbs{}}\PY{l+s+s2}{cos(x))\PYZdl{}}\PY{l+s+s2}{\PYZdq{}}
                    \PY{p}{,}\PY{l+s+s2}{\PYZdq{}}\PY{l+s+s2}{quad estimated}\PY{l+s+s2}{\PYZdq{}}\PY{p}{,}\PY{l+s+s2}{\PYZdq{}}\PY{l+s+s2}{Least squares estimated}\PY{l+s+s2}{\PYZdq{}}\PY{p}{]}\PY{p}{,} \PY{n}{loc} \PY{o}{=} \PY{l+m+mi}{0}\PY{p}{)}
         \PY{n}{fig2}
\end{Verbatim}

\texttt{\color{outcolor}Out[{\color{outcolor}26}]:}
    
    \begin{center}
    \adjustimage{max size={0.9\linewidth}{0.9\paperheight}}{Assignment3_files/Assignment3_45_0.png}
    \end{center}
    { \hspace*{\fill} \\}
    

	

	
		
    \section{Conclusions}\label{conclusions}

We observe that the fourier estimation of \(e^x\) disagrees
significantly with the function close to \(0\), but agrees almost
perfectly in the case of \(\cos(\cos(x))\). This is caused by the
presence of a discontiuity at \(x=0\) for the periodic extension of
\(e^x\). This discontiuity leads to non uniform convergence of the
fourier series, which means that the partial sums obtained using the
fourier coefficients converge at different rates for different values of
\(x\). Let us define the partial fourier summation \(S_n(x)\) of
\(f(x)\) as the sum of the first \(n\) odd and even fourier components.
Then, non uniform convergence means that, given a \(\delta > 0\), one
cannot find a single \(N \in \mathbb{N}\) for which

\[ \| S_n(x) - f(x) \| <  \delta\]

If \(n>N\) for all \(x\) in the interval of convergence.\\
This difference in the rates of convergence leads to the \textbf{Gibb's
phenomenon}, which is the ringing observed at discontiuities in the
fourier estimation of a discontiuous function. This ringing is present
for any large \(N\), but the series still converges as \(N \to \infty\).
This explains the mismatch in the fourier approximation for \(e^x\).

	


    % Add a bibliography block to the postdoc
    
    
    
    \end{document}
